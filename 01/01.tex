\documentclass[a4paper]{article} 
\addtolength{\hoffset}{-2.25cm}
\addtolength{\textwidth}{4.5cm}
\addtolength{\voffset}{-3.25cm}
\addtolength{\textheight}{5cm}
\setlength{\parskip}{0pt}
\setlength{\parindent}{0in}

%----------------------------------------------------------------------------------------
%	PACKAGES AND OTHER DOCUMENT CONFIGURATIONS
%----------------------------------------------------------------------------------------

\usepackage{blindtext} % Package to generate dummy text
\usepackage{charter} % Use the Charter font
\usepackage[utf8]{inputenc} % Use UTF-8 encoding
\usepackage{microtype} % Slightly tweak font spacing for aesthetics
\usepackage[english, ngerman]{babel} % Language hyphenation and typographical rules
\usepackage{amsthm, amsmath, amssymb} % Mathematical typesetting
\usepackage{float} % Improved interface for floating objects
\usepackage[final, colorlinks = true, 
            linkcolor = black, 
            citecolor = black]{hyperref} % For hyperlinks in the PDF
\usepackage{graphicx, multicol} % Enhanced support for graphics
\usepackage{xcolor} % Driver-independent color extensions
\usepackage{marvosym, wasysym} % More symbols
\usepackage{rotating} % Rotation tools
\usepackage{censor} % Facilities for controlling restricted text
\usepackage{listings, head/style/lstlisting} % Environment for non-formatted code, !uses style file!
\usepackage{pseudocode} % Environment for specifying algorithms in a natural way
\usepackage{head/style/avm} % Environment for f-structures, !uses style file!
\usepackage{booktabs} % Enhances quality of tables
\usepackage{tikz-qtree} % Easy tree drawing tool
\tikzset{every tree node/.style={align=center,anchor=north},
         level distance=2cm} % Configuration for q-trees
\usepackage{head/style/btree} % Configuration for b-trees and b+-trees, !uses style file!
\usepackage[backend=biber,style=numeric,
            sorting=nyt]{biblatex} % Complete reimplementation of bibliographic facilities
\addbibresource{ecl.bib}
\usepackage{csquotes} % Context sensitive quotation facilities
\usepackage[ddmmyyyy]{datetime} % Uses YEAR-MONTH-DAY format for dates
\renewcommand{\dateseparator}{.} % Sets dateseparator to '-'
\usepackage{fancyhdr} % Headers and footers
\pagestyle{fancy} % All pages have headers and footers
\fancyhead{}\renewcommand{\headrulewidth}{0pt} % Blank out the default header
\fancyfoot[L]{} % Custom footer text
\fancyfoot[C]{} % Custom footer text
\fancyfoot[R]{\thepage} % Custom footer text
\newcommand{\note}[1]{\marginpar{\scriptsize \textcolor{red}{#1}}} % Enables comments in red on margin
\usepackage{xcolor}

\definecolor{maroon}{rgb}{0.5, 0.0, 0.0}

\renewcommand{\labelenumi}{\arabic{enumi}.}

%----------------------------------------------------------------------------------------

\usepackage{enumitem, graphicx}

\begin{document}

\fancyhead[C]{}
\hrule \medskip
\begin{minipage}{0.295\textwidth} 
\raggedright
\footnotesize
Friedrich Greisiger (5722487)\hfill\\
Francisco Jose Cortes Aldaco (5978432)\hfill\\ 
Xenia Morera Martínez (5364350)\hfill\\
Lea Nickel (5153726)\hfill\\ 
Alexander Held (5943764)\hfill\\ 
\end{minipage}
\begin{minipage}{0.4\textwidth} 
\centering
\normalsize 
Machine Learning\\
\large 
Assignment 01\\ 
\normalsize 
Team: 15\\ 
\end{minipage}
\begin{minipage}{0.295\textwidth} 
\raggedleft
\today\hfill\\
\end{minipage}
\medskip\hrule 
\bigskip

%-------------------------------
%	CONTENTS
%-------------------------------

\section{General questions}
\begin{enumerate}
    \item 
    Stages: Pre-processing, Feature Extraction, Feature Selection, Model Building, Evaluation \& Model Selection, Post-processing.
    \item 
    Supervised learning: the model is trained using labeled data, where each input comes with a corresponding output or label. The goal is for the model to learn a function that maps inputs to the correct outputs, allowing it to make accurate predictions on new, unseen data.
    
    In unsupervised learning, the model is trained on data without labeled outputs. The model’s objective is to identify patterns, groupings, or structures within the data.
    \item 
    Overfitting: The model is too complex and learns noise in the training data. This will cause the model to perform well on training data but poorly on new unseen data.
    
    Underfitting: The model is too simple and fails to capture patterns. As a result it misses important relationships, which leads to poor performance on both training and new data.
\end{enumerate}



\section{Naive bayes}
\begin{enumerate}
    \item 

The probability that a car is stolen is \( P(\text{yes}) = \frac{6}{10} = 0.6 = 60\% \).

The probability that a car is not stolen is \( P(\text{no}) = \frac{4}{10} = 0.4 = 40\% \).

The probability that a car is red given it is stolen is \( P(\text{red} \mid \text{yes}) = \frac{3}{6} = 0.5 = 50\% \).

The probability that a car is a grand tourer given it is stolen is \( P(\text{grand tourer} \mid \text{yes}) = \frac{2}{6} \approx 0.33 = 33\% \).

The probability that a car is domestic given it is stolen is \( P(\text{domestic} \mid \text{yes}) = \frac{2}{6} \approx 0.33 = 33\% \).

The probability that a car is red given it is not stolen is \( P(\text{red} \mid \text{no}) = \frac{1}{4} = 0.25 = 25\% \).

The probability that a car is a grand tourer given it is not stolen is \( P(\text{grand tourer} \mid \text{no}) = \frac{2}{4} = 0.5 = 50\% \).

The probability that a car is domestic given it is not stolen is \( P(\text{domestic} \mid \text{no}) = \frac{3}{4} = 0.75 = 75\% \).

\item 

\[
Z = \sum_{k=1}^{2} P(y = k) \prod_{i=1}^{3} P(x_i | y = k)
\]

\begin{align*}
Z &= P(y = \text{yes}) \cdot P(x_1 = \text{red} | y = \text{yes}) \cdot P(x_2 = \text{grand tourer} | y = \text{yes}) \cdot P(x_3 = \text{domestic} | y = \text{yes}) \\
&\quad + P(y = \text{no}) \cdot P(x_1 = \text{red} | y = \text{no}) \cdot P(x_2 = \text{grand tourer} | y = \text{no}) \cdot P(x_3 = \text{domestic} | y = \text{no})
\end{align*}

\[
Z = (0.6 \times 0.5 \times 0.33 \times 0.33) + (0.4 \times 0.25 \times 0.5 \times 0.75)
\]

\[
Z = 0.03267 + 0.0375 = 0.07017
\]

\[
P(y = \text{yes} | x_1 = \text{red}, x_2 = \text{grand tourer}, x_3 = \text{domestic}) = \frac{P(y = \text{yes}) \prod_{i=1}^{3} P(x_i | y = \text{yes})}{Z}
\]

\[
P(y = \text{yes} | x_1 = \text{red}, x_2 = \text{grand tourer}, x_3 = \text{domestic}) = \frac{0.6 \times 0.5 \times 0.33 \times 0.33}{0.07017}
\]

\[
P(y = \text{yes} | x_1 = \text{red}, x_2 = \text{grand tourer}, x_3 = \text{domestic}) = \frac{0.03267}{0.07017} = 46.67\%
\]


    \item 

Benefits: Naive Bayes is simple, computationally efficient, and effective for high-dimensional data like text. It performs well even with small datasets and can handle irrelevant features due to the independence assumption.

Downsides: The strong independence assumption is often unrealistic, leading to poor performance with correlated features. Probability estimates may also be inaccurate, limiting its effectiveness in complex tasks.

    \item 
\end{enumerate}



\section{Ranking Losses}
\begin{enumerate}
    \item 
    \item 
\end{enumerate}



\end{document}
